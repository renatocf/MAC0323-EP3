\documentclass[a4paper,12pt]{article}

\usepackage[utf8]{inputenc}
\usepackage[brazil]{babel}

\usepackage{setspace}                   % espaçamento flexível
\usepackage{indentfirst}                % indentação do primeiro parágrafo
\usepackage[fixlanguage]{babelbib}
\usepackage[usenames,svgnames,dvipsnames]{xcolor}
\usepackage[font=small,format=plain,labelfont=bf,up,textfont=it,up]{caption}
\usepackage[a4paper,top=2.54cm,bottom=2.0cm,left=2.0cm,right=2.54cm]{geometry}

\usepackage[pdftex,plainpages=false,pdfpagelabels,pagebackref,colorlinks=true,
            citecolor=DarkGreen,linkcolor=NavyBlue,urlcolor=DarkRed,
            filecolor=green,bookmarksopen=true]{hyperref}   % links coloridos
\usepackage[all]{hypcap}                                    % soluciona o 
                                                            % problema com o 
                                                            % hyperref e capitulos
 
\title{Projeto de Iniciação Científica}
\author{Renato Cordeiro Ferreira}
\date{23/04/2012}

% ---------------------------------------------------------------------------- %
% INFORMAÇÕES
\pdfinfo{%
  /Title    (Análise de Ferramentas para Predição de DNA e RNA)
  /Author   (Renato Cordeiro Ferreira)
  /Creator  (Renato Cordeiro Ferreira)
  /Producer (Renato Cordeiro Ferreira)
  /Subject  (Proposta de Iniciação Científica)
  /Keywords (Iniciação Científica, Bioinformática, Predição Genética)
}

\begin{document}


\newpage %%%%%%%%%%%%%%%%%%%%%%%%%%%%%%%%%%%%%%%%%%%%%%%%%%%%%%%%%%%%%%%%%%%%%%%%%%%%%%%
\pagenumbering{arabic}     % começamos a numerar 

\section{Introdução} %%%%%%%%%%%%%%%%%%%%%%%%%%%%%%%%%%%%%%%%%%%%%%%%%%%%%%%%%%%%%%%%%%%

    Neste exercício-programa, tínhamos como objetivo realizar uma ação
    que usasse o \LaTeX

\section{Objetivo} %%%%%%%%%%%%%%%%%%%%%%%%%%%%%%%%%%%%%%%%%%%%%%%%%%%%%%%%%%%%%%%%%%%%%

  O projeto tem por objetivo continuar o extenso processo de validação necessária
  para a publicação do MYOP em revista de alto impacto, em particular desenvolvendo
  processos automáticos de comparação de predições em relação a dois outros preditores:
  AUGUSTUS e Genemark.HMM, com eventual desenvolvimento de novos modelos organismo-
  específicos para serem disponibilizados à comunidade científica.
  
  Para tanto, deseja-se realizar o treinamento iterativo dos 11 organismo já presentes
  com conjuntos de treinamento por conjunto validado, utilizando os programas MYOP e
  Genemark.HMM. A partir de então, criar uma comparação global entre os modelos de ambos
  e entre as predições geradas pelo MYOP e pelo AUGUSTUS a partir deles. Baseado nesta
  experiência, pretende-se desenvolver um ambiente para realização automática de testes.
  
  Num segundo momento, objetiva-se implementar um configurador gráfico para o MYOP, 
  baseado inicialmente na plataforma CoEd. % Adicionar citação Alan
  
\section{Metodologia} %%%%%%%%%%%%%%%%%%%%%%%%%%%%%%%%%%%%%%%%%%%%%%%%%%%%%%%%%%%%%%%%%
  
  Para o início do projeto, será feito um estudo básico dos conceitos de Biologia 
  Molecular envolvidos no processo de predição de genes \citep{TheCell2002}. 
  Com esta base, passaremos a avaliar os processos estatísticos e computacionais
  envolvidos na predição de genes, com a leitura de artigos sobre o assunto: \citet{Costa2003}, 
  \citet{Snyder1995}, \citet{Zhang2002}, \citet{Korf2004}, \citet{Mathe2002}, 
  \citet{Wang2004}.
  
  Em seguida, para continuar com o extenso processo de validação do Sorgo, treinaremos
  modelos para os 11 genomas-base já existetes utilizando o método iterativo para,
  então, utilizá-los nas predições realizadas do AUGUSTUS e do MYOP. Os resultados 
  obtidos serão comparados com os já existentes das predições \textsl{ab initio}.
  
  Baseado nos dados obtidos com a cana-de-açúcar, repetiremos os experimentos realizados
  sobre o genoma do \textsl{Sorghum bicolor}, com dados obtidos dos sites da NCBI
  \citep{NCBI} e do banco de dados do genoma das plantas \citep{PlantGDB}. Esta 
  espécie vegetal, filogenéticamente próxima da cana, oferece desafios parecidos, 
  e servirá para novamente comparar os desempenhos relativos entre o MYOP e o AUGUSTUS.
  
  Com base nas experiências adquiridas nestas avaliações, utilizaremos a aprendizagem da 
  linguagem Perl \citep{BeginingPerl2001, CorePerl2001} para a criação de scripts que
  automatizem estes processos. 
  
  Como bibliografia de apoio, usaremos Programming: Principles and Practice Using C++
  \citep{ProgrammingC++1994} e Design Patterns: Elements of Reusable Object-Oriented 
  Software \citep{DesignPatterns1994} para verificações em códigos C++ (ToPS) e 
  para a construção de um configurador gráfico para o MYOP.
  
\newpage % BIBLIOGRAFIA %%%%%%%%%%%%%%%%%%%%%%%%%%%%%%%%%%%%%%%%%%%%%%%%%%%%%%%%%%%%%%%
  \singlespacing   % espaçamento simples
  \bibliographystyle{plainnat-ime} % citação bibliográfica textual
  \bibliography{bibliografia}  % associado ao arquivo: 'bibliografia.bib'
  
\end{document}
